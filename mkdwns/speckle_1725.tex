% Options for packages loaded elsewhere
\PassOptionsToPackage{unicode}{hyperref}
\PassOptionsToPackage{hyphens}{url}
%
\documentclass[
]{article}
\usepackage{amsmath,amssymb}
\usepackage{iftex}
\ifPDFTeX
  \usepackage[T1]{fontenc}
  \usepackage[utf8]{inputenc}
  \usepackage{textcomp} % provide euro and other symbols
\else % if luatex or xetex
  \usepackage{unicode-math} % this also loads fontspec
  \defaultfontfeatures{Scale=MatchLowercase}
  \defaultfontfeatures[\rmfamily]{Ligatures=TeX,Scale=1}
\fi
\usepackage{lmodern}
\ifPDFTeX\else
  % xetex/luatex font selection
\fi
% Use upquote if available, for straight quotes in verbatim environments
\IfFileExists{upquote.sty}{\usepackage{upquote}}{}
\IfFileExists{microtype.sty}{% use microtype if available
  \usepackage[]{microtype}
  \UseMicrotypeSet[protrusion]{basicmath} % disable protrusion for tt fonts
}{}
\makeatletter
\@ifundefined{KOMAClassName}{% if non-KOMA class
  \IfFileExists{parskip.sty}{%
    \usepackage{parskip}
  }{% else
    \setlength{\parindent}{0pt}
    \setlength{\parskip}{6pt plus 2pt minus 1pt}}
}{% if KOMA class
  \KOMAoptions{parskip=half}}
\makeatother
\usepackage{xcolor}
\usepackage[margin = 2cm]{geometry}
\usepackage{longtable,booktabs,array}
\usepackage{calc} % for calculating minipage widths
% Correct order of tables after \paragraph or \subparagraph
\usepackage{etoolbox}
\makeatletter
\patchcmd\longtable{\par}{\if@noskipsec\mbox{}\fi\par}{}{}
\makeatother
% Allow footnotes in longtable head/foot
\IfFileExists{footnotehyper.sty}{\usepackage{footnotehyper}}{\usepackage{footnote}}
\makesavenoteenv{longtable}
\usepackage{graphicx}
\makeatletter
\def\maxwidth{\ifdim\Gin@nat@width>\linewidth\linewidth\else\Gin@nat@width\fi}
\def\maxheight{\ifdim\Gin@nat@height>\textheight\textheight\else\Gin@nat@height\fi}
\makeatother
% Scale images if necessary, so that they will not overflow the page
% margins by default, and it is still possible to overwrite the defaults
% using explicit options in \includegraphics[width, height, ...]{}
\setkeys{Gin}{width=\maxwidth,height=\maxheight,keepaspectratio}
% Set default figure placement to htbp
\makeatletter
\def\fps@figure{htbp}
\makeatother
\setlength{\emergencystretch}{3em} % prevent overfull lines
\providecommand{\tightlist}{%
  \setlength{\itemsep}{0pt}\setlength{\parskip}{0pt}}
\setcounter{secnumdepth}{-\maxdimen} % remove section numbering
\usepackage{titling}
\pretitle{\begin{left} \includegraphics[width=2in,height=2in]{/Users/acoleman/Documents/GitHub/speckle.imaging/images/Ellison Medical Institute Logo_Bronze.png}\LARGE\\}
\posttitle{\end{left}}
\usepackage{booktabs}
\usepackage{longtable}
\usepackage{array}
\usepackage{multirow}
\usepackage{wrapfig}
\usepackage{float}
\usepackage{colortbl}
\usepackage{pdflscape}
\usepackage{tabu}
\usepackage{threeparttable}
\usepackage{threeparttablex}
\usepackage[normalem]{ulem}
\usepackage{makecell}
\usepackage{xcolor}
\ifLuaTeX
  \usepackage{selnolig}  % disable illegal ligatures
\fi
\usepackage{bookmark}
\IfFileExists{xurl.sty}{\usepackage{xurl}}{} % add URL line breaks if available
\urlstyle{same}
\hypersetup{
  pdftitle={1725 Speckle Imaging},
  pdfauthor={Abby Coleman},
  hidelinks,
  pdfcreator={LaTeX via pandoc}}

\title{1725 Speckle Imaging}
\author{Abby Coleman}
\date{2025-04-15}

\begin{document}
\maketitle

{
\setcounter{tocdepth}{2}
\tableofcontents
}
\section{General Conclusions}\label{general-conclusions}

I used subsampled number of puncta as the dependent variable in my
analysis, in accordance with results of the speckle imaging optimization
experiments, which concluded this endpoint would be used for speckle
imaging going forward.

The plate containing 1725M has a negative Z'factor due to a high
variance in the DMSO - R1881 (100\% control) condition.

None of the drugs could be modeled with a log logistic curve.

Within each plate, all 1725 compounds exhibited a dose response similar
to Enzalutamide.

\emph{1725 M, R, and S exhibited similar dose response.} However,
because of the negative Z'factor on the 1725M plate and the compound's
high minimum response, results from this compound may not be accurate.

\section{Cell Count}\label{cell-count}

\begin{center}\includegraphics{speckle_1725_files/figure-latex/ccplot-1} \end{center}

The 1725M plate had the lowest distribution of cell count.

The 1725 R and S plates seem to exhibit some positional differences in
cell count.

To calculate subsampled puncta, I randomly subsampled all wells to 785
cells with the slice\_sample() function from dplyr\_1.1.4.

\newpage

\section{Z'Factor}\label{zfactor}

\begin{center}\includegraphics[width=0.5\linewidth]{speckle_1725_files/figure-latex/zprime_plots-1} \end{center}

\begin{longtable}[]{@{}lr@{}}
\toprule\noalign{}
Plate & Z'Factor \\
\midrule\noalign{}
\endhead
\bottomrule\noalign{}
\endlastfoot
1725M & -0.36 \\
1725R & 0.48 \\
1725S & 0.46 \\
\end{longtable}

1725M has a negative Z'factor, indicating that the separation band
between high and low controls is not wide enough.

\begin{center}\includegraphics[width=0.9\linewidth]{speckle_1725_files/figure-latex/wells-1} \end{center}

Looking at the raw values, you can see that the relatively high variance
in the 1725M plate's high control (DMSO - R1881) condition is what makes
this plate's separation band too small to calculate relative puncta
effectively.

\newpage

\section{Plate \%CV}\label{plate-cv}

\begin{center}\includegraphics[width=0.4\linewidth]{speckle_1725_files/figure-latex/cv_plots-1} \end{center}

Plate \%CV's are low for all plates.

\newpage

\section{Dose Response}\label{dose-response}

None of the dose response curves were able to fit a log logistic curve
of any parameterization. I will use splines to compare dose response for
1725M, R, and S.

\begin{center}\includegraphics{speckle_1725_files/figure-latex/analysis-1} \end{center}

Within each plate, the 1725 compounds respond comparatively with
Enzalutamide.

\begin{center}\includegraphics{speckle_1725_files/figure-latex/drcs_1725s-1} \end{center}

The 1725 compounds respond similarly across doses. 1725M has a high
minimum response, but this plate also had a negative Z'factor.

\end{document}
